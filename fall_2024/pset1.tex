\documentclass[12pt,answers]{exam}
\usepackage{amsfonts, amsmath, amssymb, amsthm, fullpage, mathtools, tikz, enumitem}
\linespread{1.2}
\usepackage[ruled,vlined]{algorithm2e}

\title{CSE 598, Fall 2024, Homework \#1}
\date{}
\author{Instructor: Zilin Jiang}

\usepackage{tikz}
\usetikzlibrary{quantikz}

\newtheorem{theorem}{Theorem}
\newtheorem{corollary}[theorem]{Corollary}
\newtheorem{lemma}[theorem]{Lemma}
\newtheorem{observation}[theorem]{Observation}
\newtheorem{proposition}[theorem]{Proposition}
\newtheorem{definition}[theorem]{Definition}
\newtheorem{claim}[theorem]{Claim}
\newtheorem{fact}[theorem]{Fact}
\newtheorem{assumption}[theorem]{Assumption}

\theoremstyle{remark}
\newtheorem*{remark}{Remark}
\newtheorem*{example}{Example}

%%%%%%%%%%% MACROS (ADD YOUR OWN BELOW) %%%%%%%%%
\newcommand{\abs}[1]{\left\lvert #1 \right\rvert}
\newcommand{\norm}[1]{\left\lVert #1 \right\rVert}
\newcommand{\inprod}[1]{\left\langle #1 \right\rangle}
\newcommand{\R}{\mathbb{R}}
\newcommand{\C}{\mathbb{C}}
\newcommand{\ip}[2]{\langle #1 | #2 \rangle}
\newcommand{\ph}{\varphi}
\newcommand{\eps}{\varepsilon}
\newcommand{\aal}{\abs{\alpha}^2}
\newcommand{\bbe}{\abs{\beta}^2}
\newcommand{\stdb}{\{\ket0,\ket1\}}
\newcommand{\buv}{\{\ket{u},\ket{v}\}}
\newcommand{\tO}{\widetilde{O}}
\newcommand{\cvec}[2]{\begin{bmatrix}#1 \\ #2\end{bmatrix}}
\newcommand{\rvec}[2]{\begin{bmatrix}#1 & #2\end{bmatrix}}

\begin{document}

\maketitle

\paragraph{Question 2(b)} Given $\eps > 0$, pick a natural number $n$ such that $1/2^n < \eps$ and $2^n - 1$ is divisible by $3$. This can be achieved by taking $n$ to be a sufficiently large even number. The idea is to pick a number $n$ from a set $S$ of $2^n$ numbers uniformly at random. We split the set $S$ into three sets of size $1$, $(2^n-1)/3$ and $2(2^n-1)/3$. Depending on which set contains $n$, our program outputs FAILURE, $0$ or $1$ respectively. Therefore our subroutine goes as follows.

\begin{algorithm}
    \SetAlgoLined
    \KwOut{$r$}
    $n = 0$\;
    \For{$i \leftarrow 1$ \KwTo $n$}{
        $b \leftarrow 0 \text{ or } 1$ equally likely\;
        $n \leftarrow 2n + b$\;
    }
    \tcc{Now $n$ is a random $n$-digit binary number.}
    \eIf{$n = 0$}{
        $r \leftarrow $FAILURE\;
    }{
        \eIf{$n \% 3 = 0$}{
            $r \leftarrow 0$\;
        }{
            $r \leftarrow 1$\;
        }
    }
    \caption{Simulating a biased coin}
\end{algorithm}

\paragraph{Question 3(c)} Expressing the two operators in $\mathbb{Z}/2\mathbb{Z}$, we have $\mathrm{NOT}(x) = 1 + x$ and $x_1 \mathrm{XOR} x_2 = x_1 + x_2$. Thus a Boolean formula with variables $x_1, \dots, x_n$ built with NOT and XOR only can be rewritten as $c + \sum_{i\in I} x_i$ for some $c \in \{0,1\}$ and some $I \subseteq \{1,2,\dots, n\}$. It is easy to see that such a Boolean function is either constant, or balanced (meaning, assumes 0 or 1 equally likely). However, one can easily construct infinitely many Boolean functions (or equivalently, truth tables) that are neither constant nor balanced.

\paragraph{Question 4(b)} The algorithm $C''$ repeatedly runs $C$ until it outputs $0$ or $1$, and then $C''$ returns the last output. The number of runs of $C$ follows the geometric distribution with success probability $p = 10\%$. Therefore the average (or mean) of the number of runs is $1/p = 1/10\% = 10$.

\paragraph{Question 4(c)} Let $n$ be a natural number to be determined later. The algorithm $C'$ repeatedly runs $C$ for up to $n$ times, and $C'$ returns $0$ as soon as $C$ outputs $0$; otherwise $C'$ return $1$. Since $C$ has no false negatives, when $f(x) = 1$, $C$ always outputs $1$, and so $C'$ always outputs $1$, which implies that $C'$ has no false negatives, either. When $f(x) = 0$, the probability that $C'$ outputs $1$ is equal to the probability that $C$ outputs $1$ for $n$ runs, which is equal to $(90\%)^n$. To ensure that the failure probability of $C'$ is below $2^{-500}$, it suffices to take $n = \lceil -500 / \log_2 90\%\rceil = 3290$.

\paragraph{Question 4(d)} Let $n$ be an odd number to be determined later. The algorithm $C'$ repeatedly runs $C$ for $n$ times, and returns the majority of the outputs. Suppose $f(x) = 1$. Then the probability that $C'(x) = 0$ is equal to the probability that more than $n/2$ outputs of $C$ are $0$. Let $X$ be the sum of all the $n$ outputs of $C$. The average of $X$, denoted by $\mu$, is $0.6n$. The Chernoff bound for Bernoulli random variable $X$ says that $\Pr(X \le (1-\delta)\mu) \le e^{-\delta^2 \mu/2}$. Thus $\Pr(X \le 0.5n) \le e^{-(1/6)^2(0.6n)/2} = e^{-n/120}$. Due to symmetry, when $f(x) = 0$, the probability that $C'(x) = 1$ can also be bounded by $e^{-n/120}$. To ensure that the failure probability of $C'$ is below $2^-500$, it suffices to take $n = 120 \times 500 \times \ln 2 = 41,589$.

\end{document}
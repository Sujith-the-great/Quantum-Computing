\documentclass[12pt,answers]{exam}
\usepackage{amsfonts, amsmath, amssymb, amsthm, fullpage, mathtools, tikz, enumitem}
\linespread{1.2}
\usepackage[ruled,vlined]{algorithm2e}

\title{CSE 598, Fall 2024, Homework \#2}
\date{}
\author{Instructor: Zilin Jiang}

\usepackage{tikz}
\usetikzlibrary{quantikz}

\newtheorem{theorem}{Theorem}
\newtheorem{corollary}[theorem]{Corollary}
\newtheorem{lemma}[theorem]{Lemma}
\newtheorem{observation}[theorem]{Observation}
\newtheorem{proposition}[theorem]{Proposition}
\newtheorem{definition}[theorem]{Definition}
\newtheorem{claim}[theorem]{Claim}
\newtheorem{fact}[theorem]{Fact}
\newtheorem{assumption}[theorem]{Assumption}

\theoremstyle{remark}
\newtheorem*{remark}{Remark}
\newtheorem*{example}{Example}

%%%%%%%%%%% MACROS (ADD YOUR OWN BELOW) %%%%%%%%%
\newcommand{\abs}[1]{\left\lvert #1 \right\rvert}
\newcommand{\norm}[1]{\left\lVert #1 \right\rVert}
\newcommand{\inprod}[2]{\left\langle #1 \mid #2 \right\rangle}
\newcommand{\R}{\mathbb{R}}
\newcommand{\C}{\mathbb{C}}
\newcommand{\ip}[2]{\langle #1 | #2 \rangle}
\newcommand{\ph}{\varphi}
\newcommand{\eps}{\varepsilon}
\newcommand{\aal}{\abs{\alpha}^2}
\newcommand{\bbe}{\abs{\beta}^2}
\newcommand{\stdb}{\{\ket0,\ket1\}}
\newcommand{\pmbs}{\{\ket+,\ket-\}}
\newcommand{\buv}{\{\ket{u},\ket{v}\}}
\newcommand{\bs}[2]{\{#1,#2\}}
\newcommand{\tO}{\widetilde{O}}
\newcommand{\cvec}[2]{\begin{bmatrix}#1 \\ #2\end{bmatrix}}
\newcommand{\rvec}[2]{\begin{bmatrix}#1 & #2\end{bmatrix}}

\begin{document}

\maketitle

\paragraph{Question 2(e)} Let $U' = \sum_{i} \ket{\phi_i}\bra{\psi_i}$. For every $j$, we have $U'\ket{\psi_j} = \sum_{i} \ket{\phi_i}\inprod{\psi_i}{\psi_j} = \ket{\phi_j}$. Thus $U$ and $U'$ agree on the $\psi$-basis, and so $U = U'$.

\begin{remark}
    One needs to mention that $U$ and $\sum_{i} \ket{\phi_i}\bra{\psi_i}$ are equal because they agree on a basis of $\C^d$.
\end{remark}

\paragraph{Question 3} Let $n$ be a large integer depending on $\eps$ to be chosen later. Set $\theta = \pi/(2n)$. The idea is to perform a series of $n$ measurements where the bases are
$$
  \bs{R_\theta\ket0}{R_\theta\ket1}, \bs{R_{2\theta}\ket0}{R_{2\theta}\ket1}, \dots, \bs{R_{n\theta}\ket0}{R_{n\theta}\ket1},
$$
where $R_?$ is the rotation transformation with a specific angle. After the first the measurement, we have the mixed state
$$
\begin{cases}
  R_{\theta}\ket0 & \text{ with probability } \cos^2\theta;\\
  R_\theta\ket1   & \text{ with probability } \sin^2\theta.
\end{cases}
$$

Consider the mixed state after the second measurement. If the state is $R_{\theta}\ket0$ after the second measurement, then the mixed state after the second measurement is
$$
\begin{cases}
  R_{2\theta}\ket0 & \text{ with probability } \cos^2\theta;\\
  R_{2\theta}\ket1 & \text{ with probability } \sin^2\theta.
\end{cases}
$$
If the state is $R_{\theta}\ket1$ after the second measurement, then the mixed state after the second measurement is
$$
\begin{cases}
  R_{2\theta}\ket0 & \text{ with probability } \sin^2\theta;\\
  R_{2\theta}\ket1 & \text{ with probability } \cos^2\theta.
\end{cases}
$$
Compounding the probabilities, overall the mixed state after the second measurement is
$$
\begin{cases}
  R_{2\theta}\ket0 & \text{ with probability } \cos^2\theta\cos^2\theta + \sin^2\theta\sin^2\theta;\\
  R_{2\theta}\ket1 & \text{ with probability } 2\sin^2\theta\cos^2\theta.
\end{cases}
$$
In particular, after the second measurement, the probability that the state is $R_{2\theta}\ket0$ is at least $\cos^4\theta$. Similarly, we can prove that after the $k$th measurement, the probability that the state is $R_{k\theta}\ket0$ is at least $\cos^{2k}\theta$. When $k = n$, the output state is $R_{n\theta}\ket{0} = \ket1$ with probability at least
$$
  \cos^{2n}\theta = \cos^{2n}\left(\frac{\pi}{2n}\right).
$$
Recall that $\cos\theta = 1 - 2\sin^2(\theta/2)$. Since $\theta = \pi/(2n) \in [0, \pi/2]$, and so $\sin(\theta/2) \in (0,\theta/2)$, we have
$$
  \cos\theta = 1 - 2\sin^2(\theta/2) > 1 - 2(\theta/2)^2,
$$
which implies that
$$
  \cos^{2n}\theta > \left(1-\frac{\theta^2}{2}\right)^n = \left(1-\frac{\pi^2}{8n^2}\right)^n.
$$
One can use standard tools from calculus to show that the last expression tends to $1$ as $n$ approaches infinity. Thus it is possible to choose $n$ large enough so that the probability of outputting $\ket1$ is at least $1 - \eps$.

\begin{remark}
  There is an adaptive strategy (which consists of several runs where each run might depend on previous ones). Measure in the $\pmbs$ basis and then measure in the $\stdb$ basis. The output is $\ket0$ and $\ket1$ equally likely. When the readout is $\ket1$, we output the qubit immediately. Otherwise, we repeat the process. Every iteration, we output $\ket1$ with probability $1/2$. As long as $2^{-n} < \eps$, up to $n$ iterations suffice.
\end{remark}

\paragraph{Question 6(b)} By applying a CNOT gate with the second qubit as the control, and then a Hadamard gate on the second qubit, one can work out that the output state will be transformed to on of the following:
\[
  \ket{00}, \ket{01}, \ket{10}, \ket{11}.
\]
Now Bob only need to measure in the standard basis to tell Alice's message.

\begin{remark}
    If the CNOT gate is applied with the first qubit as the control, and then a Hadamard gate on the first qubit, the output state will be $\ket{vu}$ rather than $\ket{uv}$.
\end{remark}

\end{document}